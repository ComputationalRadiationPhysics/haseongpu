\section{Introduction}
\numberwithin{equation}{section}

Any excited upper laser state will inevitably emit spontaneously photons due
its radiative life time. Those photons freely travel within the gain medium
itself and experience amplification. This process can drastically reduce the
stored energy and is called Amplified Spontaneous Emission (ASE). ASE does not
only reduce locally stored energy, it dislocates energy into not pumped areas,
like sourrounding claddings or mounts. Therefore it has to be considered as an
important factor for heat distribution inside and outside of the pumped laser
gain medium.

Together with laser induced damage and thermal related issues is ASE one of the
challenges for every laser gain medium design to overcome. The impact of ASE on
energy storage has been studied since the early days of lasers \cite{Intro1,Intro2,Intro3}.
However, with the increasing importance of diode pumped lasers, this particular
topic came in the focus of interest again \cite{Intro4}.

For most of the applied cases ASE and related problems cannot be treated
analytically and is therefore subject to numerical analysis. Most numerical
codes usually simplify the problem in order to gain computation time. With the
advent of GPU based codes, it is of strong interest to take advantage of the
good scalability of the Monte-Carlo approach used in our recent model \cite{Intro4}.
In this paper we concentrate on the adaptation, extention and performance
increase of the model described and benchmarked in \cite{Intro4}.
