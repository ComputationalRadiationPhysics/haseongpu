\section{Introduction}
\numberwithin{equation}{section}
Any excited upper laser state will inevitably emit spontaneously
photons due its radiative life time. Those photons travel freely
within the gain medium itself and experience amplification.  This
process, called Amplified Spontaneous Emission (ASE), drastically
reduces the stored energy ASE does not only affect locally stored
energy, but also dislocates energy into non-pumped areas, like
surrounding claddings or mounts. Therefore, it is an important factor
for heat distribution inside and outside of the pumped laser gain
medium. Together with laser induced damage and thermal related issues,
ASE is one of the main challenges for every laser gain medium design
to overcome.

The impact of ASE on energy storage has been studied since the early
days of lasers~\cite{Intro1,Intro2,Intro3}. However, with the
increasing importance of diode pumped lasers, this particular topic
came in the focus of interest again~\cite{Intro4}.

For most of the applied cases, ASE cannot be treated analytically and
therefore needs to be numerically modeled. In order to improve
computation time, numerical codes usually use simplified
models. However, with the advent of massively parallel accelerator
hardware, it becomes possible to solve the full problem by exploiting
the strong scalability of existing Monte-Carlo algorithms used in our
recent model \cite{Intro4}. In this paper we concentrate on the
adaptation, extention and performance increase of the model described
and benchmarked in~\cite{Intro4}.
