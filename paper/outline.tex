\documentclass[a4paper,11pt]{article}

\title{Parallel ASE-Flux}

\begin{document}
\maketitle
\begin{abstract}
This is the abstract.
\end{abstract}

\section{Introduction}
\begin{itemize}
  \item ASE flux from a physics point of view
  \item simulation takes a lot of time
  \item highly parallel problem can be solved with GPUs
\end{itemize}


\section{Methods}
\subsection{general ideas}

\subsubsection{the crystal mesh}
\begin{itemize}
  \item crystal is sampled and meshed into prisms
  \item \textbf{Image: meshing of the crystal}
\end{itemize}

\subsubsection{ray tracing}
\begin{itemize}
  \item ray propagates through the mesh
  \item gain is calculated during the propagation
  \item \textit{(formula for gain calculation)}
  \item \textbf{Image: 2D propagation of ray through triangle structure}
\end{itemize}

\subsubsection{Monte Carlo Simulation}
\begin{itemize}
  \item \textit{integral is expressed by formula} (see Daniel's thesis)
  \item MonteCarlo simulation as a way to calculate the integral for the whole crystal
\end{itemize}

\subsubsection{Reflections}
\begin{itemize}
  \item depending on the material and coating, there are different reflectivities
  \item only reflections on upper and lower plane
\end{itemize}

\subsection{parallel ideas}
there are several ways to exploit parallelization:
\begin{itemize}
  \item samplepoints ($\rightarrow$ implemented as MultiGPU)
  \item rays ($\rightarrow$ implented per device through threads/blocks)
  \item wavelengths
\end{itemize}
\textit{formula for the additive monte carlo simulation}


\section{acceleration techniques}
ways to improve the speed
\subsection{Importance Sampling}
\begin{itemize}
  \item is used to start many rays only in the "interesting" areas for every samplepoint
  \item standard technique
\end{itemize}

\subsection{adaptive number of rays}
\begin{itemize}
  \item since most samples behave in a good way, most samples don't need many rays (low MSE).
  \item some outliers have high Mean Square Error
  \item use more rays to address the problem $\rightarrow$ takes a lot more time
  \item adaptive rays: check MSE after the calculation and restart with more rays if required
\end{itemize}


\section{octrace tool}
\textbf{really include this section with all the detail?}
\subsection{features}
\begin{itemize}
  \item polychromatic (multiple wavelengths)
  \item reflections
  \item variable + adaptive number of rays
  \item multigpu
  \item threshold for MSE
  \item creation of VTK files containing $\Phi_{ASE}$, reflections per prism etc.
  \item comparison with existing VTK files 
  \item wrapper script for matlab
\end{itemize}

\subsection{usage}
\subsubsection{input}
\textit{format for the input files:}
\begin{itemize}
  \item points
  \item triangles
  \item sigma\_A
  \item sigma\_E
  \item \dots
\end{itemize}

\subsubsection{output}
\textit{format of the output files:}
\begin{itemize}
  \item Phi\_ASE
  \item expectation
  \item N\_rays
\end{itemize}


\section{Results}
\subsection{comparison of values}
\begin{itemize}
  \item comparison with previous values from Daniel's thesis
  \item (comparison with results from an experiment)
  \item \textbf{Image: gain VS time for one single point. Overlay between experiment/daniels sim/our sim}
\end{itemize}

\subsection{runtimes}
\begin{itemize}
  \item adaptive number of rays MUCH faster than fixed number of rays
  \item use MSE to adjust the precision of the simulation rather than increasing the default number of rays
  \item law of diminishing returns: to lower the MSE below a certain threshold, exponentially more rays are required
  \item \textbf{Image: runtime VS precision of simulation} (precision based on MSE)
  \item very low computation times result in values comparable to extremly long simulations.
  \item distributing the computation to multiple devices scales well
  \item \textbf{Image: MultiGPU with different number of devices VS time}
\end{itemize}


\end{document}
