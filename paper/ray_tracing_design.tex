\subsection{Design}
\begin{itemize}
  \item Mesh stucture
    \begin{itemize}

      \item The crystal is seen as a 2D plane that gets sampled by a set of
        points. 

      \item Delauny triangulation is used to connect these points to a mesh of
        triangles.

      \item The triangles are extruded by a thickness to generate a slice of
        prisms. This slice gets duplicated several times to cover the whole
        crystal.

      \item The sample points can be distributed non-uniformly to increase
        spatial resolution in areas of interest.

    \end{itemize}

  \item Ray tracing on prism structure\\
    \textbf{Picture: 2d prism structure and one ray cutting through.}

    \begin{itemize}
      
      \item To calculate the amplification of a single ray from a position $r$
        to a sampled point $r_0$, the ray is traced along its path through the
        prisms. The intersections with prism surfaces define line segments of a
        certain length. These segments are used for gain calculation.

      \item Starting from point $r$ inside a prism $p$, there are 5 possible
        intersection planes for the ray. One plane for the top and bottom
        surfaces and one for each of the 3 horizontal sides of $p$.

      \item Based on the intersection plane, the next prism is known due to a
        neighboring relation in the datastructure.  neighbor relation, that you
        store in a list.

      \item The intersection is calculated for each prism along the path, until
        $r_0$ is reached.

    \end{itemize}
  \item Parallel ray tracing
    \begin{itemize}
      \item Every ray can be calculated independently from each other.
        Exploiting this parallelism provides a great opportunity to boost
        performance. Only the resulting gains for each sample point have to be
        combined in the end.
      \item GPUs are a suitable tool to realize the parallelisation
      \item Mention mersenne twister
    \end{itemize}
\end{itemize}
