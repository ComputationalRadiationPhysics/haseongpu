\section{Methods to reduce the number of rays}

\subsection{Importance sampling}
\begin{itemize}
\item Importance sampling is a well known technique in the domain
  of statistics\textbf{[insert src]}. In case of ASE calculation a presampling of
  a sample point is done to figure out important areas
  inside the mesh. Important areas generate more rays for a
  particular sample point.
\item Importance sampling can increase the efficiency of Monte-Carlo-Simulations (reduce variance)
\end{itemize}

\subsection{Adaptive rays}
\begin{itemize}
\item Usage of Mean squared error (MSE).
     \[f(\vec{r_0}) = \frac{1}{n} \sum_{i=1}^n g_i \]
     \[f^2(r_0) = \frac{1}{n} \sum_{i=1}^n g_i^2 \]
     \[MSE(r_0) = \sqrt{\frac{f^2(r_0) - f(r_0)^2}{n}}\]
\item Dependant on a $MSE$-threshold and the $MSE$ value
  of the sample, the number of rays per sample point 
  will be increased to a maximum number.
\item Thus not every sample point needs to be sampled
  with a high number of rays to obtain high precision.
  Only sample points with $MSE(r_0) \quad \textgreater \quad MSE$-threshold need
  resampling. This saves a lot of calculation time.
\item \textbf{insert graphic here}\\
  X = max-$MSE$, Y = t in s or Rays total
\end{itemize}
