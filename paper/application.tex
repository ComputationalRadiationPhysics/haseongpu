\section{Application}
\numberwithin{equation}{section}
\label{sec:application}
The application is available as a commandline-tool and reads its
simulation data from plain text files. It was improved over the  orginal application
from D.~Albach~\cite{ASE2010} by adding
polychromatic wavelengths, adaptive sampling, repetitive sampling,
multi-GPU computation and reflections.


For an easy access, the offered MATLAB compatible interface
(see~\ref{label:matlab_interface}) is highly recommended. Experienced
users with the intention to call the application directly should
consult the README file, the provided example code or the source code
for more details.

\subsection{Installation and Usage}
The application was tested and developed under a linux environment and
runs on NVIDIA graphics cards with compute capability 2.0 (Fermi
generation) or higher.  It can be built by running \texttt{make} inside the
application directory, provided that \texttt{make} (tested with 3.82), \texttt{gcc} (up
to 4.6.2) and CUDA (5.0) are installed.  This will also setup a MATLAB
file calcPhiASE.m in the application directory.

To run the application from MATLAB, 2 steps are necessary in
advance:
\begin{enumerate}
\label{label:matlab_interface}
  \item Include calcPhiASE.m from the application directory into your
    MATLAB path
  \item Call calcPhiASE function from the MATLAB
    script: \[[phiASE,~MSE,~nRays] = calcPhiASE(args)\]
\end{enumerate}

\subsubsection{Input Arguments}
An overview of the necessary input arguments is given in the list
below:

\begin{description}
\label{label:input}
  \item[Mesh information]\mbox{}
    \begin{itemize}
      \setlength{\itemindent}{-2.5em}
    \item Structure (points, triangles, prisms)
    \item Thickness of a mesh slice
    \item Number of planes
    \item Refractive index of mesh and surroundings
    \item Reflectivities of mesh planes bottom and top
    \end{itemize}

  \item[Properties] \mbox{}
    \begin{itemize}
      \setlength{\itemindent}{-2.5em}
      \item Stimuli in sample points and volume/prisms $\beta$
      \item Crystal Fluorescence $\tau_{f}$
      \item Cladding (use of different materials)
      \item Doping of the active gain medium $N_{tot}$
    \end{itemize}

  \item[Laser information] \mbox{}
    \begin{itemize}
      \setlength{\itemindent}{-2.5em}
      \item Absorption cross section $\sigma_a$
      \item Emission cross section $\sigma_e$
    \end{itemize}

   \item[Algorithm information] \mbox{}
    \begin{itemize}
      \setlength{\itemindent}{-2.5em}
       \item Maximal number of rays for adaptive sampling
       \item MSE-threshold for adaptive sampling
       \item Number of repetitions for repetitive sampling
       \item Decide whether to activate reflections
     \end{itemize}
    
 \end{description}
A detailed description about the MATLAB compatible interface and the
required arguments are given in the README file and should be
consulted.

\subsubsection{Output}
The simulation returns 3 data sets:
\begin{description}
  \item[phiASE] is a matrix of $\Phi_{ASE}(s_i)$
  \item[MSE] is a matrix of reached $MSE(s_i)$
  \item[nRays] is a matrix of used rays for the Monte Carlo experiment
    for each sample point
    
\end{description}
The returned variables are represented as two-dimensional matrices in
which columns are slice indices and rows are point indices.  For
example, the matrix for the used gain medium has 321 rows and 10
columns, what multiplies to 3210 sample points. The value for the
$i$-th point and the $j$-th slice can then be obtained by MATLAB with:
\[value~=~values(i,j)\]
