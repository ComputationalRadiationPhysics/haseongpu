\section{Application}
\numberwithin{equation}{section}

The application is available as a commandline-tool and reads
its simulation data from plain text files. It was improved by polychromatic
wavelengths, adaptive sampling, repetetive sampling, multigpu computation and 
reflections towards the orginal application from D. Albach \cite{ASE2010}.

For an easy access the offered MATLAB compatible interface (see \ref{label:matlab_interface}) 
is highly recommended. Experienced users with the intention to call the application
directly, should consult the README file, the provided 
example code or the source code for more details.

\subsection{Installation and Usage}
The application was tested and developed under a linux environment
and runs on Nvidia graphics cards with capability 2.0 (fermi generation) or higher.
It can be built by running make inside the application directory, provided
that make (tested with 3.82), gcc (up to 4.6.2) and cuda (5.0) are installed. 
This will also setup a MATLAB file calcPhiASE.m in the application
directory. 

To run the application from MATLAB, two steps are
necessary in advance:
\begin{enumerate}
  \label{label:matlab_interface}
  \item Include calcPhiASE.m from the application directory into your MATLAB path
  \item Call calcPhiASE function from the MATLAB script: \[[phiASE,~MSE,~nRays] = calcPhiASE(args)\]
\end{enumerate}
A detailed description about the MATLAB compatible interface
is given in the README file and should be consulted.

%\subsection{MATLAB interface}
%\label{label:matlab_interface}
%Because MATLAB is widely used in academic and research institutions, a MATLAB
%interface is provided. 

%\subsubsection{Usage}
%In \ref{label:input}, a brief overview about the
%arguments of the calcPhisAse function can be found. See the README file for 
%more interface details. Otherwise it is just a function call from
%a MATLAB script:


%% \subsubsection{Input Arguments}
%% Mesh information needs to be generated in advance
%% to save calculation time, because usually $\Phi$-ASE calculations
%% run over several timesteps on the same mesh. Thus the following
%% data need to be provided by the MATLAB script:
%% \begin{itemize}
%%   \label{label:input}
%%   \item Mesh information
%%     \begin{itemize}
%%     \item Set of triangles $\mathcal{T}$
%%     \item Normal vectors of prism (triangle) sides
%%     \item Center points of the triangles
%%     \item Surfaces of triangles
%%     \item Neighbor relation between triangles $n : \mathcal{T} \times \mathcal{T}$
%%     \item Thickness of a mesh slice 
%%     \item Number of slices
%%     \item Refractive index of mesh and volume around
%%     \item Reflectivities of mesh planes bottom and top
%%     \end{itemize}

%%   \item Properties 
%%     \begin{itemize}
%%       \item Stimuli in the sample points and volume/prisms $\beta$
%%       \item Crystal Fluorescence of the active gain medium $\tau_{f}$
%%       \item Cladding of the mesh (use of different materials)
%%       \item Doping of the active gain medium $N_{tot}$
%%     \end{itemize}

%%   \item Laser information
%%     \begin{itemize}
%%       \item laser absorption $\sigma_a$
%%       \item laser emission $\sigma_e$
%%     \end{itemize}

%%   \item Algorithm information
%%     \begin{itemize}
%%       \item Maximal number of rays for adaptive sampling
%%       \item MSE-threshold for adaptive sampling
%%       \item Decide whether to activate reflections
%%     \end{itemize}
    
%% \end{itemize}

%% \subsubsection{Output}
%% The MATLAB function returns data 
%% for each sample point as follows:
%% \begin{itemize}
%%   \item $\Phi_{ASE}$ values
%%   \item MSE values
%%   \item Number of Rays of Monte Carlo simulation
%% \end{itemize}
