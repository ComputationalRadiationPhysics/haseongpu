\section{Application}
\numberwithin{equation}{section}

The application is available as a commandline-tool and reads
its simulation data from plain text files. 
For an easy access the offered MATLAB interface (see \ref{label:matlab_interface}) 
is highly recommended. Experienced users can also call the application
directly. Please consult the README file, the provided 
example code or the source code for more details.\\
\textbf{TODO}: improvment features
\subsection{Features}
\begin{itemize}
  \item polychromatic (multiple wavelengths)
  \item reflections on top and bottom of amplifying medium
  \item variable + adaptive number of rays per sample point
  \item multigpu on one computation node
  \item threshold for MSE
 \end{itemize}

\subsection{Installation}
The application was tested and developed under a linux environment.
It can be built by running make inside the application directory, provided
that make (tested with 3.82), gcc (up to 4.6.2) and cuda (5.0) are installed. 
This will also setup a MATLAB file calcPhiASE.m in the application
directory. The device code runs on Nvidia graphics cards with 
capability 2.0 (fermi generation) or higher. 

\subsection{MATLAB interface}
\label{label:matlab_interface}
Because MATLAB is widely used in academic and research institutions, a MATLAB
interface is provided. To run the application from MATLAB, two steps are
necessary in advance:
\begin{enumerate}
  \item Include calcPhiASE.me from the application directory into your MATLAB path
  \item Call calcPhiASE function from the MATLAB script
\end{enumerate}

\subsubsection{Usage}
In \ref{label:input}, a brief overview about the
arguments of the calcPhisAse function can be found. See the README file for 
more interface details. Otherwise it is just a function call from
a MATLAB script:
\[[phiASE,~MSE,~nRays] = calcPhiASE(args)\]

\subsubsection{Input Arguments}
Mesh information needs to be generated in advance
to save calculation time, because usually $\Phi$-ASE calculations
run over several timesteps on the same mesh. Thus the following
data need to be provided by the MATLAB script:
\begin{itemize}
  \label{label:input}
  \item Mesh information
    \begin{itemize}
    \item Set of triangles $\mathcal{T}$
    \item Normal vectors of prism (triangle) sides
    \item Center points of the triangles
    \item Surfaces of triangles
      %\item Set of triangle points $\mathcal{P}$
    \item Neighbor relation between triangles $n : \mathcal{T} \times \mathcal{T}$
    \item Thickness of a mesh slice 
    \item Number of slices
    \item Refractive index of mesh and volume around
    \item Reflectivities of mesh planes bottom and top
    \end{itemize}

  \item Constants 
    \begin{itemize}
      \item Stimulus in the sample points ($\beta-cells$)
      \item Stimulus in the volume/prisms ($\beta-volume$)
      \item Crystal Fluoresncence of the active gain medium
      \item Cladding of the mesh (use of different materials)
      \item Allocation of the active gain medium
    \end{itemize}

  \item Laser information
    \begin{itemize}
      \item laser absorption $\sigma_a$
      \item laser emission $\sigma_e$
    \end{itemize}

  \item Algorithm information
    \begin{itemize}
      \item Maximal number of rays for adaptive sampling
      \item MSE-threshold for adaptive sampling
      \item Decide whether to activate reflections
    \end{itemize}
    
\end{itemize}

\subsubsection{Output}
\tablehead{\hline \textbf{Output} & \textbf{Description} \\\hline \hline}
\begin{supertabular}{| p{3cm} | p{4cm} |}
  \hline
  phiASE & solution of simulation for each sample point \\\hline
  MSE & Mean squared error for each sample point \\\hline
  nRays & number of rays per sample point \\\hline
\end{supertabular}
