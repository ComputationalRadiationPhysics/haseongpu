\section{Application}
The application is available as a commandline-tool and reads
its simulation data from plain text files (in future also from stdin).
For an easy access the offered MATLAB interface (see \ref{label:matlab_interface}) 
is highly recommended. Experienced users can also call the application
directly. Please consult the README file, the provided 
example code or the source code for more details.\\
\textbf{TODO:}~octave interface ?

\subsection{Installation}
The application was tested and developed under a linux environment.
It can be built by running make inside the application directory, provided
that make (up to 3.81), gcc (up to 4.6.2) and cuda (5.0) are installed. 
This will also setup a MATLAB file calcPhiASE.m in the application
directory. The device code runs on Nvida graphics cards from compute 
capability 2.0 (at least fermi generation). 

\subsection{MATLAB interface}
\label{label:matlab_interface}
Because MATLAB is widely used in academic and research institutions, a MATLAB
interface is provided. You need to include the MATLAB file calcPhisASE.m from
the application directory into your MATLAB path and call the calcPhiASE function 
from you MATLAB environment.

\subsubsection{Usage}
In \ref{label:input} you'll find a brief overview about the
arguments of the calcPhisAse function. See the README file for 
more interface details. Otherwise it's just a function call from
your MATLAB script.
\[[phiASE,~MSE,~nRays] = calcPhiASE(args)\]
Mesh information need to be generated in advanced
to save calculation time, because usually $\Phi$-ASE calculations
run over several timesteps on the same mesh. 


%%\subsection{Features}
%%\begin{itemize}
%%  \item polychromatic (multiple wavelengths)
%%  \item reflections on top and bottom of amplifying medium
%%  \item variable + adaptive number of rays per sample point
%%  \item multigpu on one computation node
%%  \item threshold for MSE
%%  \item creation of VTK files containing $\Phi_{ASE}$, reflections per prism etc.
%%  \item comparison with existing VTK files 
%% \end{itemize}



\subsubsection{Input Arguments}
\label{label:input}
\tablehead{\hline \textbf{Argument} & \textbf{Description} \\\hline \hline}
\begin{supertabular}{| p{3cm} | p{4cm} |}
  points & Sample points of the mesh \\\hline
  triangleNormalsX & X coordinates of triangle edges normal vector \\\hline
  triangleNormalsY & Y coordinates of triangle edges normal vector \\\hline
  forbiddenEdges & Adjacent edges of neighbor triangles\\\hline
  triangleNormalPoint & Start points of triangle edges normal vector \\\hline
  triangleNeighbors & Neighbor relation of triangles \\\hline
  trianglePoints & Points of the triangle vertices \\\hline
  thickness & Thickness of one mesh level \\\hline
  numberOfLevels & Total number of levels \\\hline
  nTot & \textbf{TODO} Some constant \\\hline
  betaVol & \textbf{TODO} Some constant for each sample Points \\\hline
  laserWavelength & \textbf{TODO} Laserparameter \\\hline
  crystal & \textbf{TODO}Crystalparameter \\\hline
  prismBetaValues & \textbf{TODO} Some constant for every prism of the mesh \\\hline
  triangleSurfaces & Surfaces of triangles \\\hline
  triangleCenterX & X coordinate of triangle center point\\\hline
  triangleCenterY & Y coordinate of triangle center point\\\hline
  clad & \textbf{TODO} Cladding stuff \\\hline
  cladNum & \textbf{TODO} More Cladding stuff \\\hline
  cladAbs & \textbf{TODO} More Cladding stuff \\\hline
  refractiveIndices &  Refraction index of mesh and volume around \\\hline
  reflectivities & Reflectivities of mesh planes top and bottom \\\hline
  maxRays & Maximal number of rays for adaptive sampling \\\hline
  MSEThreshold & MSE-Threshold for adaptive sampling \\\hline
  useReflections & Switch to activate reflections \\\hline
\end{supertabular}

\subsubsection{Output}
\tablehead{\hline \textbf{Output} & \textbf{Description} \\\hline \hline}
\begin{supertabular}{| p{3cm} | p{4cm} |}
  \hline
  phiASE & solution of simulation for each sample point \\\hline
  MSE & Mean squared error for each sample point \\\hline
  nRays & number of rays per sample point \\\hline
\end{supertabular}
