\section{Ray tracing}
\subsection{Design}
\begin{itemize}
\item Mesh stucture
  \begin{itemize}
  \item The crystal plane is divided into a set of points and 
    these points are connected to a mesh of triangles by 
    Delaunay triangulation.
  \item The triangles will be extruded by a thickness to generate prisms.
    These Prisms will be duplicated by several levels to fill the 
    whole crystal volume.			  
  \end{itemize}
\item Ray tracing on prism structure
  \begin{itemize}
  \item To calculate amplification of the spontaneous emission of a ray from
    point r to r0, we need to know which prisms are on this path and how
    long are the lines of intersection in this prisms.
  \item Starting from point r, the ray has 5 possible planes for 
    intersection inside the prism. There are 2 planes of prism-surfaces
    and 3 planes of prism-sides.
  \item Depending on intersected plane, you know the next prism by a neighbor
    relation, that you store in a list. Besides it's forbidden to return 
    to the last prism.
  \item You repeat this until you reach point r0.
    
  \end{itemize}
\item Parallel ray tracing
  \begin{itemize}
  \item Every ray can be calculated independently from each other. So it's
    a perfect chance to boost the performance by parallelization. 
    The ray gain values have to be added atomically to the ase value
    of point r0.
  \item Used parallel architecture
  \item Mention mersenne twister
  \end{itemize}
\end{itemize}
\subsection{Implementation}
\begin{itemize}
\item Different wavelength
\item Reflections
\end{itemize}
