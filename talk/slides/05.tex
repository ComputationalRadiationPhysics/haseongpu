\autobookmark
\begin{frame}[t]{Monte Carlo can be implemented with ray tracing}
  \myonly{1}{\includegraphics[height=0.6\paperheight]{graphics/monte_carlo_description_2.pdf}}
  \myonly{2}{\includegraphics[height=0.6\paperheight]{graphics/monte_carlo_description_3.pdf}}
  \myonly{3}{\includegraphics[height=0.6\paperheight]{graphics/monte_carlo_description_4.pdf}}
%
%  \begin{columns}[T]
%
%    \begin{column}{.5\textwidth}
%      \textbf{graphic: ASE crystal, with a photon travelling through it?}\\[2ex]
%      \textbf{alternative: laser setup, pumping a gain medium?}
%    \end{column}
%
%    \begin{column}{.5\textwidth}
%      \begin{itemize}
%      \myuncover{1}{3}{
%        \item In laser physics, gain media are pumped by lasers\\[1ex]
%      }
%      \myuncover{2}{3}{
%        \item Over time, electrons fall back into lower energy levels, causing
%          the spontaneous emission of a photon\\[1ex]
%      }
%      \myuncover{3}{3}{
%        \item Photon travels through the medium, where it is amplified by more
%          photons that are released
%      }
%  \end{itemize}
%    \end{column}
%
%  \end{columns}
\end{frame}
